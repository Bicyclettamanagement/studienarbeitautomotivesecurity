%%%%%%%%%%%%%%%%%%%%%%%%%%%%%%%%%%%%%%%%%%%%%%%%%%%%%%%%%%%%%%%%%%%%%%%%%%%%%%
%% Descr:       Vorlage für Berichte der DHBW-Karlsruhe, Datei mit Abkürzungen
%% Author:      Prof. Dr. Jürgen Vollmer, vollmer@dhbw-karlsruhe.de
%% $Id: abk.tex,v 1.4 2017/10/06 14:02:03 vollmer Exp $
%% -*- coding: utf-8 -*-
%%%%%%%%%%%%%%%%%%%%%%%%%%%%%%%%%%%%%%%%%%%%%%%%%%%%%%%%%%%%%%%%%%%%%%%%%%%%%%%

\chapter*{Abkürzungsverzeichnis}                   % chapter*{..} -->   keine Nummer, kein "Kapitel"
						         % Nicht ins Inhaltsverzeichnis
% \addcontentsline{toc}{chapter}{Akürzungsverzeichnis}   % Damit das doch ins Inhaltsverzeichnis kommt

% Hier werden die Abkürzungen definiert
\begin{acronym}[DHBW]
  % \acro{Name}{Darstellung der Abkürzung}{Langform der Abkürzung}
  %SICK
 \acro{Abk}[Abk.]{Abkürzung}
 \acro{GBC}[GBC]{Global Business Center}
 \acro{CD}[CD]{Corporate Department}
 \newacroplural{CD}[CD]{Corporate Departments}
 \acro{DMT}[DMT]{Digital Manufacturing Team}
 \acro{ISN}[ISN]{Intelligent Supply Network}
 \acro{BU}[BU]{Business Unit}
 
 %Automotive
 \acro{ECU}[ECU]{Electronic Control Unit}
 \newacroplural{ECU}[ECUs]{Electronic Control Units}
 \acro{CAN}[CAN]{Controller Area Network}
 \acro{DLC}[DLC]{Data Link Connector}
 
 %Testing
 \acro{ISTQB}[ISTQB]{International Software Testing Qualification Board}
 \acro{SUT}[SUT]{System under Test}
 
 %Allgemein
 \acro{HTML}[HTML]{Hypertext Markup Language}
 \acro{UI}[UI]{User Interface}
 \acro{API}[API]{Application Programming Interface}
 \acro{MVC}[MVC]{Model View Controller}

 \acro{IDE}[IDE]{Integrated Development Environment}
 % Folgendes benutzen, wenn der Plural einer Abk. benöigt wird
 % \newacroplural{Name}{Darstellung der Abkürzung}{Langform der Abkürzung}
 \newacroplural{Abk}[Abk-en]{Abkürzungen}

 \acro{H2O}[\ensuremath{H_2O}]{Di-Hydrogen-Monoxid}

 % Wenn neicht benutzt, erscheint diese Abk. nicht in der Liste
 \acro{NUA}{Not Used Acronym}
\end{acronym}
