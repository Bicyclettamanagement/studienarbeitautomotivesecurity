\chapter{Einführung}

Autos stellen einen sehr großen Anteil der Infrastruktur heutzutage dar. In einer Umfrage im Jahr 2022 gaben über 70 Prozent der Befragten an, ein eigenes Auto zu besitzen \cite[vgl.][]{Statista.2022}. Unzählige Autos sind täglich auf den Straßen unterwegs. Im Zuge der Digitalisierung werden moderne Autos zunehmend mit neuen Features und Technologien ausgestattet, mit dem Ziel, die Bedienung des Fahrzeugs möglichst komfortabel zu gestalten. Das Auto nimmt der fahrenden Person immer mehr Aufgaben ab, wie zum Beispiel das Abblenden, Einparken oder im Fall von selbst-fahrenden Autos sogar das Steuern des Fahrzeugs an sich. Zudem steigt die Anzahl der Entertainmentfeatures, wie zum Beispiel das Verbinden eines Mobiltelefons mit dem Fahrzeug. Ein Effekt dieser Entwicklung ist, dass zum einen die einzelnen Fahrzeugteile intern zunehmend miteinander vernetzt werden. Zum anderen steigt aber auch die Relevanz der Kommunikation des Fahrzeugs mit externen Systemen. Insgesamt sind die elektronischen Systeme in heutigen Fahrzeugen deutlich komplexer und bieten mehr Schnittstellen als noch vor 20 Jahren. Diese zunehmende Komplexität schafft neue Angriffsflächen für Cyberangriffe. Experimente in der Vergangenheit wie zum Beispiel von Charlie Miller und Chris Valasek \cite[vgl.][]{Greenberg.2015} haben jedoch bereits gezeigt, dass die Sicherheitsmaßnahmen der Automobilhersteller oft nicht ausreichen, um die Fahrzeuge zuverlässig gegen solche Angriffe zu schützen.

\section{Motivation}
Eines der schockierendsten Ereignisse der letzten Jahre im Bereich der Automotive Cyber Security war die oben erwähnte Aktion von Miller und Valasek im Jahr 2015 \cite[vgl.][]{Greenberg.2015}. Den beiden Hackern gelang es, einen Jeep Cherokee über das Internet zu kompromittieren. Dabei verschafften sie sich nicht nur Zugriff zur grundlegenden Board-Elektronik wie dem Radio oder den Scheibenwischern, sondern es gelang ihnen auch, die Bremsen und den Motor zu deaktivieren. Sie konnten das Fahrzeug fernsteuern und der eingeweihte Fahrer war ihnen hilflos ausgeliefert. Dieses Experiment fand natürlich nur zu Forschungs- und Demonstrationszwecken statt. Aktionen wie diese zeigen jedoch anschaulich, wozu eine Person mit böswilligen Absichten theoretisch in der Lage wäre. Sicherheitslücken wie diese können schlimmstenfalls zum Verlust von Menschenleben führen. Aus diesem Grund ist es wichtig, das dem Thema der Automotive Security noch mehr Aufmerksamkeit gewidmet wird. Hersteller müssen sich intensiver mit den durch die zunehmende Vernetzung der Autos entstandenen Angriffsmöglichkeiten beschäftigen und Sicherheitslücken bestenfalls präventiv, ansonsten so schnell wie möglich, schließen. Daher widmet sich diese Arbeit diesen besagten Angriffsmöglichkeiten.

\section{Zielsetzung}
Diese Arbeit soll einen Überblick über die Angriffsflächen eines Automobils sowie über einige Lösungsansätze für diese Schwachstellen schaffen. Hierzu erfolgt zunächst eine Erläuterung der notwendigen theoretischen Grundlagen wie dem Aufbau des internen Netzwerks eines Automobils sowie notwendigen Grundlagen der Cyber Security. Anschließend sollen die verschiedenen Angriffsmöglichkeiten eines Autos aufgezeigt werden. Darauf folgt die Sammlung und Evaluierung von Schutzmaßnahmen gegen diese Angriffsmöglichkeiten mit Blick auf die Frage, wo die Hersteller ansetzen können oder müssen, um ihre Autos sicherer zu gestalten.