\documentclass[
   ngerman          % neue deutsche Rechtschreibung
  ,a4paper          % Papiergrösse
% ,twoside          % Zweiseitiger Druck (rechts/links)
% ,10pt             % Schriftgrösse
% ,11pt
  ,12pt
  ,pdftex
%  ,disable         % Todo-Markierungen auschalten
]{article}

\usepackage[utf8]{inputenc}
%\usepackage{dhbwVorlageStyle}
\usepackage{graphicx}
\usepackage[onehalfspacing]{setspace}

\title{Studienarbeit Automotive Security \\ --- \\ Grobkonzeption \vspace{4cm}}
\author{
        Jonas Kölblin \\
		Kurs: TINF20B5 \\
		Betreuer: Ralf Brune \\
		Studiengang: Allgemeine Informatik \\
        Duale Hochschule Baden-Württemberg Karlsruhe        
}
\date{\today}



\begin{document}
\begin{figure}[t]
\flushright
\includegraphics[width=4cm]{dhbw-logo}
\end{figure}

\maketitle


%\begin{abstract}
%This is the paper's abstract \ldots
%\end{abstract}
\newpage
%\tableofcontents
%\newpage
\section{Einführung}
\subsection{Hinführung}
Einstieg in das Thema
\subsection{Motivation}
Relevanz: fortschreitende Vernetzung, mangelhafte Sicherung, Konsequenzen einer Schwachstelle
\subsection{Zielsetzung}
Netzwerkaufbau erläutern, Angriffsflächen aufzeigen, Schutzmöglichkeiten erläutern und bewerten

\section{Grundlagen}
\subsection{Automotive}
Aufbau des Netzwerks in Fahrzeugen
\subsection{Cybersecurity}
Cybersecurity Grundlagen (Kryptographie etc.)


\section{Ausarbeitung}
\subsection{Überblick Angriffsflächen}
Welche Angriffsflächen gibt es?
\subsection{Schutzmaßnahmen}
Sammlung und Erklärung von Schutzmaßnahmen gegen die Angriffsmöglichkeiten mit Blick auf Effektivität und Wirtschaftlichkeit, Kann man sich überhaupt gegen alle Angriffsmöglichkeiten schützen?

\section{Fazit}
\subsection{Schlussfolgerung}
Zusammenfassung und Bewertung der Ergebnisse
\subsection{Ausblick}
Ausblick auf weitere Entwicklung der Thematik

\newpage
\section{Bisherige Literatur}
Manuel Wurm: Automotive Cybersecurity (2022) \\
Tobias Brennich \& Martin Moser: Putting Automotive Security to the Test (2020) \\
Charlie Miller: Lessons learned from hacking a car (2019) \\
Stephen Checkoway et al: Comprehensive Experimental Analyses of Automotive Attack Surfaces (2011) \\
Ishtiaq Rouf et al: Security and Privacy Vulnerabilities of In-Car Wireless Networks: A Tire Pressure  Monitoring System Case Study (2010) \\
Guillermo A. Francia Automotive Vehicle Security Metrics (2021)
Andreea-Ina Radu \& Flavio Garcia: LeiA: A Lightweight Authentication Protocol for CAN (2016) \\
Van Huynh Le et al: Security and privacy for innovative automotive applications: A survey (2018) \\
...

%\bibliographystyle{abbrv}
%\bibliography{simple}

\end{document}
This is never printed
