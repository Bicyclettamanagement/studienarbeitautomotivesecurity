\chapter{Grundlagen}
In diesem Kapitel sollen die für das weitere Verständnis notwendigen theoretischen Grundlagen erläutert werden. Dazu gehört zunächst der Aufbau des Netzwerks in einem Fahrzeug. Des Weiteren werden relevante Grundlagen der Cyber Security erklärt.

\section{Automotive Networking}
Im Inneren von Autos befinden sich heutzutage eine Vielzahl elektronischer Systeme, von denen jedes mit benachbarten Komponenten kommunizieren kann. Die einzelnen elektronischen Systeme werden als \acp{ECU} bezeichnet. Moderne Autos enthalten in der Regel über 50 verschiedene \acsp{ECU} \cite[vgl.][6]{Miller.2013}. Da diese Kontrolleinheiten zum Teil lebensentscheidende Aufgaben übernehmen, muss die Kommunikation zwischen den Einheiten möglichst in Echtzeit erfolgen. \\

\begin{figure}[H]
\centering
\includegraphics[width=\textwidth]{communication-protocols}
\label{fig:communication-protocols}
\caption{Verschiedene Kommunikationsprotokolle in Automobil-Netzwerken}
\quelle{\cite[2]{MohammadAshjaei.2021}}
\end{figure}

Für die Venetzung der \acsp{ECU} kommen verschiedene Technologien zum Einsatz (siehe Abbildung \ref{fig:communication-protocols}). Diese werden im Folgenden genauer erläutert. Die relevanteste davon ist im Automotive-Bereich der sogenannte \acs{CAN}-Standard.

\subsection{Controller Area Network}
Die elektronischen Kontrolleinheiten eines Autos sind typischerweise über einen oder mehrere Busse, die auf dem \ac{CAN}-Standard basieren, miteinander verbunden. Hierbei kommunizieren die \acsp{ECU} über \acs{CAN}-Pakete. Diese werden an alle Komponenten gesendet, welche dann jeweils basierend auf dem Inhalt entscheiden, ob das Paket für sie bestimmt ist oder nicht. Eine Identifikation der Quelle oder Authentisierung gibt es in diesem Standard nicht. \cite[vgl.][7]{Miller.2013} \\
In Abbildung \ref{fig:canbus-ford2010} ist das \acs{CAN}-Netzwerk eines 2010 Ford Escape dargestellt. Das abgebildete Netzwerk verfügt über zwei Busse, einen medium speed (MS) und einen high speed (HS) \acs{CAN}-Bus. Beide Busse enden hier im \ac{DLC} (siehe Kapitel \ref{OBD-II}).
In Automotive Netzwerken lassen sich zwei Arten von \acs{CAN}-Paketen finden: normale \acs{CAN}-Pakete und diagnostische \acs{CAN}-Pakete. 

\begin{figure}[h]
\centering
\includegraphics[width=\textwidth]{Miller_canbus-ford2010}
\label{fig:canbus-ford2010}
\caption{Beispiel des \acs{CAN}-Netzwerks eines 2010 Ford Escape}
\quelle{\cite[19]{Miller.2013}}
\end{figure}

\subsubsection{Normale \acs{CAN}-Pakete}
Normale Pakete werden von \acsp{ECU} gesendet und können entweder Informationen oder Befehle enthalten. Typischerweise werden sie alle Millisekunden gesendet. Auf Anwendungsebene enthalten die \acs{CAN}-Pakete einen Identifier, die zu übertragenden Daten und manchmal noch eine Prüfsumme, um sicherzustellen, dass das Paket korrekt übertragen wurde. Der Identifier gibt sowohl an, für welche \acsp{ECU} das Paket bestimmt ist, als auch, welche Priorität das Paket hat. \cite[vgl.][9]{Miller.2013}

\subsubsection{Diagnostische \acs{CAN}-Pakete}
Diagnostische Pakete tauchen während des normalen Betriebs des Autos im Normalfall nicht auf. Sie werden von Diagnose-Werkzeugen gesendet, die beispielsweise von Mechanikern genutzt werden um mit den \acsp{ECU} im Auto zu kommunizieren. So können Mängel und Fehlfunktionen entdeckt oder andere Informationen gewonnen werden. Das Format von diagnostischen \acs{CAN}-Paketen ähnelt dem von normalen Paketen, erfolgt jedoch meist nach strengeren Konventionen. Standards hierfür sind zum Beispiel ISO-TP, ISO 14229 und ISO 14230. \cite[vgl.][10]{Miller.2013}



\subsection{Schnittstellen}
\subsubsection{OBD-II Port} \label{OBD-II}

\section{Cyber Security}