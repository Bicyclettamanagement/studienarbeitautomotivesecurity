\chapter{Einführung}

Autos stellen einen sehr großen Anteil der Infrastruktur heutzutage dar. In einer Umfrage im Jahr 2022 gaben über 70 Prozent der Befragten an, ein eigenes Auto zu besitzen \cite[vgl.][]{Statista.2022}. Unzählige Autos sind täglich auf den Straßen unterwegs. Im Zuge der Digitalisierung werden moderne Autos zunehmend mit neuen Features und Technologien ausgestattet, mit dem Ziel, die Bedienung des Fahrzeugs möglichst komfortabel zu gestalten. Das Auto nimmt der fahrenden Person immer mehr Aufgaben ab, wie zum Beispiel das Abblenden, Einparken oder im Fall von selbst-fahrenden Autos sogar das Steuern des Fahrzeugs an sich. Zudem steigt die Anzahl der Entertainmentfeatures, wie zum Beispiel das Verbinden eines Mobiltelefons mit dem Fahrzeug. Ein Effekt dieser Entwicklung ist, dass um Einen die einzelnen Fahrzeugteile intern zunehmend miteinander vernetzt werden. Zum anderen steigt aber auch die Relevanz der Kommunikation des Fahrzeugs mit externen Systemen. Insgesamt sind die elektronischen Systeme in heutigen Fahrzeugen deutlich komplexer und bieten mehr Schnittstellen als noch vor 20 Jahren. Diese zunehmende Komplexität schafft neue Angriffsflächen für Cyberangriffe. Experimente in der Vergangenheit wie zum Beispiel von Charlie Miller und Chris Valasek \cite[vgl.][]{Greenberg.2015} haben gezeigt, dass die Sicherheitsmaßnahmen der Automobilhersteller oft nicht ausreichen, um die Fahrzeuge zuverlässig gegen solche Angriffe zu schützen.

\section{Motivation}
Cyberangriffe sind verheerend.